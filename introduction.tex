\section{Introduction}\label{section:introduction}
In high-energy hadronic collisions, heavy quarks, i.e charm and beauty are mainly produced in hard parton scattering processes. Due to their large masses, their production cross sections can be calculated in the framework of perturbative Quantum Chromodynamics (pQCD) down to low transverse momenta~\cite{Kniehl:2008zza,Cacciari:2003uh,Kniehl:2005mk,Cacciari:2003zu}, %making them a very important tool to study 
in hadronic collisions. In heavy-ion collisions at ultra-relativistic energies, it is well established that a strongly coupled quark gluon-plasma (QGP) is formed~\cite{Karsch:2006xs,Borsanyi:2010cj,Bazavov:2011nk}. In the presence of the QGP a suppression of the heavy-flavour particles produced in the collisions with a high transverse momentum ($p_{\textrm{T}}$) is observed ~\cite{ALICE:2012ab,Abelev:2012qh,Adam:2015nna,Adam:2016khe,Sirunyan:2017xss, Acharya:2020lgn}. 
%{\color{blue}Write a sentence about v2...}

Measurements of heavy flavour hadron production in proton-proton collisions at the Large Hadron Collider (LHC) provide a way to test pQCD calculations at the highest available collision energies and constitute a baseline for the study of heavy-flavour production in heavy-ion collisions. The inclusive production cross sections of charm mesons and their decay products measured in pp collisions at the LHC at both mid-and forward-rapidity~\cite{Abelev:2012pi,Abelev:2012vra,Abelev:2012xe,Aaij:2016jht} are described by theoretical predictions based on pQCD calculations, with the collinear factorisation approach at next-to-leading order (e.g. in the general-mass variable-flavour-number scheme, GM-VFNS~\cite{Kniehl:2008eu,Kniehl:2011bk}) or at fixed order with next-to-leading-log resummation (FONLL~\cite{Cacciari:2012ny}) within theoretical uncertainties.
%The measured D-meson production cross sections in pp collisions at the LHC can also be described by pQCD calculations performed in the framework of $k_{\textrm{T}}$-factorisation in the leading order (LO) approximation~\cite{Maciula:2013oba}. 
Beauty production cross section measurements in pp collisions at $\sqrt{s}=7$ and $2.76$ TeV~\cite{Abelev:2012gx,Adam:2016wyz,ATLAS:2013cia,Khachatryan:2010yr,Aaij:2010gn}  are well described by implementations of FONLL and GM-VFNS~\cite{Cacciari:2012ny,Kniehl:2005ej,Maciula:2013wg,Abelev:2012gx,Abelev:2012sca,ATLAS:2013cia,Khachatryan:2010yr,Aaij:2010gn,Kniehl:2011bk}. 
%{\color{blue}Include HFE as well??}

In proton-nucleus collisions, the so-called ‘Cold Nuclear Matter’ (CNM) effects occur due to the presence of a nucleus in the colliding system, and, possibly, to the large density of produced particles. In particular, the parton distribution functions (PDFs) of nucleons bound in nuclei are modified with respect to those of free nucleons, which can be described by phenomenological parameterisations referred to as nuclear PDFs (nPDFs)~\cite{Eskola:2009uj,deFlorian:2003qf,Hirai:2007sx}. When the production process is dominated by gluons at low Bjorken-x, the nucleus can be described by the Colour-Glass Condensate (CGC) effective theory as a coherent and saturated gluonic system~\cite{Fujii:2013yja,Tribedy:2011aa,Albacete:2012xq,Rezaeian:2012ye}. The kinematics of the partons in the initial state can be affected by multiple scatterings 
%(transverse momentum broadening, or $k_{\textrm{T}}$ broadening)
~\cite{Lev:1983hh,Kopeliovich:2002yh} or by gluon radiation (energy loss) before or after the heavy-quark pair is produced~\cite{Vitev:2007ve}. Measurement of heavy-flavour production in \pPb collisions at the LHC will allow a study of the above mentioned effects. Previous measurements of the nuclear-modification factor of heavy-flavour hadrons and its decay leptons in \pPb collisions at $\sqrt{s_{\textrm{NN}}}=5.02$ TeV indicate no significant modification of their yields due to CNM effects in the measured  transverse momentum region within the uncertainties of the measurements~\cite{Adam:2015qda,Adam:2016wyz,Abelev:2014hha}. 


Recent measurements of light-flavour and heavy-flavour hadrons in high multiplicity pp, p-A and d-A collisions at different energies have revealed strong flow-like effects in these small systems~\cite{Acharya:2018dxy,Abelev:2012ola,Aaboud:2016yar,Chatrchyan:2013nka,ABELEV:2013wsa,Khachatryan:2014jra,Adare:2013piz,Adamczyk:2015xjc,Adare:2015ctn,Khachatryan:2010gv}. The origin of these phenomena is debated, and several models with microscopic and macroscopic approaches describe qualitatively the observed features in high multiplicity events.  While macroscopic models incorporate hydrodynamical evolution of the system~\cite{Werner:2010ss,Deng:2011at,Werner:2013ipa}, the others include overlapping strings~\cite{Bierlich:2014xba}, string percolation~\cite{Bautista:2015kwa}, multi-parton interactions and color reconnection~\cite{Sjostrand:2014zea,Ortiz:2013yxa}. A multiphase transport model~\cite{Koop:2015wea}, as well as the fragmentation of saturated gluon states~\cite{Schlichting:2016sqo,Schenke:2016lrs}, is able to describe some features of the data. The measurement of heavy-flavour production in small systems as a function of the charged-particle multiplicity produced in the collision could thus provide further insight into the processes occurring in the collision at the partonic level and the interplay between the hard and soft mechanisms in particle production in pp and \pPb collisions. 

Measurements of open and hidden charm and beauty production~\cite{Adam:2015ota,Acharya:2020pit,Adam:2018jmp,Chatrchyan:2013nza,Acharya:2020giw,Adam:2016mkz} indicate an increase of heavy-flavour production with charged-particle multiplicity measured at midrapidity. D-meson~\cite{Adam:2015ota} and J$/\psi$~\cite{Acharya:2020pit,Adam:2018jmp} production normalized to their corresponding averages in minimum bias events, measured as a function of normalized event multiplicities in pp collisions $\sqrt{s} = 13$ TeV  by the ALICE collaboration at the LHC and at $\sqrt{s} = 0.2$ TeV by the STAR collaboration at RHIC shows a stronger than linear increase. Measurement of $\Upsilon$(nS) production in pp collisions at $\sqrt{s} = 2.76$ TeV by the CMS Collaboration at midrapidity indicate a linear increase with the event activity, when measuring it at forward rapidity, and a stronger than linear increase with the event activity measured at midrapidity~\cite{Chatrchyan:2013nza}. In \pPb collisions, the relative D-meson yield increases with a faster-than-linear trend as a function of the relative charged-particle multiplicity at midrapidity and is consistent with a linear growth for multiplicity measured at large rapidity~\cite{Adam:2016mkz}. The normalised J$/\psi$ yield at larger rapidities also exhibit an increase with increasing normalised charged-particle pseudorapidity density, where the yield at backward rapidity grows faster than the forward rapidity one~\cite{Acharya:2020giw}. A possible correlation with the event multiplicity (and event shape) was also observed for the inclusive charged-particle production~\cite{Acharya:2019mzb}, and for identified particles, including multi-strange hyperons~\cite{Acharya:2018orn}.

In this paper, we present the production cross section of electrons from heavy-flavour hadron decays at midrapidity in pp collisions at $\sqrt{s}=13$ TeV and \pPb collisions at $\sqrt{s_{\textrm{NN}}}=8.16$ TeV. The cross section of electrons from heavy-flavour hadron decays is measured in the transverse momentum (\pt) down to 0.2 GeV/$c$ and up to 35 GeV/$c$ in pp collisions, which is the lowest and highest $\pt$-reach attained with the ALICE detector. %{\color{blue} Mention the low B field and cite the XeXe paper which will be published soon.} 
Results of $R_{\rm{pPb}}$ of electrons from heavy-flavour hadron decays at midrapidity are reported here, which represents the first ALICE measurement of open heavy-flavour particles in \pPb collisions at $\sqrt{s_{\textrm{NN}}}=8.16$ TeV. The relative yields of electrons from heavy-flavour hadron decays measured for the first time as a function of charged-particle multiplicity estimated at midrapidity  (\etaless 1.0) in pp and \pPb collisions are also reported. 

The paper is structured as follows. In Sec. 2, the
ALICE apparatus, its main detectors and the data samples used for the analysis are reported. In Sec. 3, the definition of multiplicity and calculation of charged-particle pseudorapidity density are presented. Sec. 4 describes the procedure employed to obtain heavy-flavour decay electron sample. Sec. 5 describes the systematic uncertainties associated with the measurements. The results of the analysis are presented and discussed in Sec. 6. Finally the paper is briefly summarised in Sec. 7.
%The extrapolation of the cross section of heavy-flavour decay electron to $\sqrt{s}=8.16$~TeV, used as a reference for the p-Pb measurement is presented in Sec. 6.

%In proton-proton collisions it is also important to consider that at high collision energies there is a substantial contribution from Multi-Parton Interactions (MPI)~\cite{}, where several interactions on the parton level can occur in a single pp collision.  At the high center-of-mass energies reached at the LHC, a substantial contribution of MPI on a harder scale can introduce a correlation between heavy-flavour particle production and the total charged particle multiplicity~\cite{}. Measurements by the CMS Collaboration of jet and underlying event properties as a function of multiplicity in pp collisions at $\sqrt{s}=7$ TeV can be better described by event generators including MPI~\cite{}. The analysis of minijet production performed by the ALICE Collaboration~\cite{} indicates that high multiplicities in pp collisions are reached through a high number of MPIs and a higher than average number of fragments per parton. the LHCb Collaboration reported measurements of double charm production in pp collisions at the LHC ($\textrm{D}^{0}+X$, $J/\psi+X$ and $J/\psi+J/\psi$, where $X = \textrm{D}^{0}, \textrm{D}^{+}, \textrm{D}^{0}_{s}, \Lambda^{+}_{c}$), which suggest that MPIs also play a role at the hard momentum scale relevant for cc production~\cite{}. 


%Recently, the study of heavy-flavour production as a function of the multiplicity of charged particles produced in the collision has attracted growing interest. Such measurements probe the interplay between hard and soft mechanisms in particle production. At LHC energies, the multiplicity dependence of heavy-flavour production is likely to be affected by the larger amount of gluon radiation associated with short-distance production processes, as well as by the contribution of Multiple-Parton Interactions (MPI)~\cite{}. It has also been argued that, due to the spatial distribution of partons in the transverse plane, the probability for MPI to occur in a pp collision increases towards smaller impact parameters~\cite{}. This effect might be further enhanced by quantum-mechanical fluctuations of gluon densities at small Bjorken-x~\cite{}.

%The measurements of prompt D mesons, inclusive and non-prompt J/ψ in pp collisions at √ s = 7 TeV [30, 31], and of the three ϒ states in pp collisions at √ s = 2.76 TeV [32], provide evidence for a similar increase of open and hidden heavy-flavour yields as a function of charged-particle multiplicity. These results suggest that the enhancement probably originates in short-distance production processes, and is not influenced by hadronisation mechanisms. The enhancement is quantitatively described by calculations including MPI contributions, namely percolation model estimates [33, 34], the EPOS 3 event generator [35, 36] and PYTHIA 8.157 calculations [37].


%Heavy quarks (charm and beauty), produced in the initial stages of hadronic collisions in hard scattering processes, provide an important testing ground for perturbative QCD calculations. Measurements of their production as a function of the charged-particle multiplicity in pp and p-Pb collisions have recently gained interest for investigating the interplay between hard and soft mechanisms of particle production. In the p-Pb collision system, the formation and the kinematic properties of heavy-flavour hadrons can be influenced at all stages  by Cold Nuclear Matter (CNM) effects and by concurrent Multiple Parton Interactions (MPI).\\
