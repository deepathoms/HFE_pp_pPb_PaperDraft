\section{Summary}\label{section:summary}
Heavy-flavour production at midrapidity was studied using electrons from heavy-flavour hadron decays in pp collisions at $\sqrt{s}=13$ TeV and in \pPb collisions at \sqrtsNN=8.16 TeV with the ALICE detector at the LHC. The \pt differential production cross-section of electrons from heavy-flavour hadron decays in pp collisions was measured in the range of 0.2 $<p_{\rm T}<$ 35 GeV$/c$ and compared with Fixed-Order-Next-to-Leading-Log pQCD calculations.
%, which and shows a good agreement within the statistical and systematic uncertainties. 
{The data lies on the upper edge of the theoretical uncertainties.} The measurements in pp collisions are an important test of pQCD calculations at the highest energy measured. The \pt differential production cross-section of electrons from heavy-flavour hadron decays in \pPb collisions was measured in the range of 0.5 $<p_{\rm T}<$ 26 GeV$/c$. The nuclear modification factor in \pPb collisions, $R_{\rm{pPb}}$, was calculated and found to be consistent with unity within the statistical and systematic uncertainties. The $R_{\rm{pPb}}$ at \sqrtsNN = 8.16 TeV was also found to be consistent with that measured at \sqrtsNN = 5.02 TeV. The $R_{\rm{pPb}}$ results shows no hot-QCD effects in the measured \pt range and provides important inputs for cold nuclear matter calculations at \sqrtsNN = 8.16 TeV. 

The multiplicity dependent production of electrons from heavy-flavour hadron decays was measured using the self-normalized yield  as a function of normalized charged particle pseudorapidity density at midrapidity in pp and \pPb collisions as a function of transverse momentum. A faster than linear growth was observed in both pp and p--Pb collisions, similar to the observation for inclusive charged particles and several other charm and beauty hadrons. While in p-Pb collisions no \pt dependence is observed within uncertainties, in pp collisions a strong \pt dependence is seen with high \pt electrons showing a faster increase as a function of normalized multiplicity. The measurement of self-normalized yield of electrons from heavy-flavour hadron decays in pp collisions was compared with PYTHIA 8.2 simulations, which reproduces the trend qualitative, but overestimates the slope at high \pt. While strong conclusions are difficult to make due the contributions of jets from hard-scattering process to the measured multiplicity at midrapidity, the self-normalized yield measurements as a function of normalized multiplicities provides more insight into the production of heavy-flavour hadrons and its interplay with charged-particle production.  Disentangling the production of heavy-flavour hadrons and the charged particle multiplicity will be an important step towards shedding light on the mechanism of production and hadronization of charm and beauty quarks in high-multiplicity environment in pp and \pPb collisions. 

