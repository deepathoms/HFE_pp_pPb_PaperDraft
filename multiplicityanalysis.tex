\section{Multiplicity definition and corrections}\label{section:multiplicityanalysis}

The production of electrons from heavy-flavour hadron decays was investigated as a function of charged-particle pseudorapidity density (\dnchdeta) in pp and \pPb collisions, using the self-normalised yield of electrons from heavy-flavour hadron decays  $\left( \dndeta/\left<\dndeta\right>\right)$ as a function the normalised charged-particle density $\left( \dnchdeta/\left<\dnchdeta\right> \right)$.  The \dnchdeta was measured in the pseudorapidity range \etaless 1.0. The \dnchdeta is evaluated using the number of tracklets (\ntracklets)~\cite{ALICE:2012xs,Acharya:2018egz}, defined as the track segments formed by joining
 pairs of hits in the two layers of the SPD pointing to the primary vertex. For this purpose, only events with a primary vertex position within \zvertexless {10} are selected to minimise non-uniformities in the SPD acceptance. 

The number of raw tracklets (\ntracklets) in an event are corrected ($N^{\rm{corr}}_{\rm{tracklets}}$) for variation of the detector conditions with time (fraction of active SPD channels) and its limited acceptance as a function of \zvertex using a data-driven event-by-event correction, following the procedure discussed in ~\cite{Abelev:2012rz,Adam:2015ota,Acharya:2020pit,Acharya:2020giw}. The correction is done by applying a \zvertex and time-dependent correction factor such that the measured average multiplicity is equalized to a reference value, which was chosen to be the largest mean SPD tracklet multiplicity observed over time. The correction factor for each event is randomly smeared using a Poisson distribution to take into account event-by-event fluctuations. The
events are sliced in $N^{\rm{corr}}_{\rm{tracklets}}$ intervals, which are  corrected for the trigger and vertex finding efficiencies. The former is estimated from Monte Carlo simulations and the latter with a data driven approach. They are below unity only for the low-multiplicity events.

Detector inefficiencies, production of secondary particles due to interactions with the detector material and particle decays lead to a difference between the number of reconstructed tracklets and the
true primary charged-particle multiplicity $N_{\rm{ch}}$~\cite{Adam:2015pza}. Monte Carlo (MC) simulations using PYTHIA8.2~\cite{Sjostrand:2006za} and DPMJET~\cite{Roesler:2000he} event generator, for pp and \pPb collisions respectively, and the GEANT 3~\cite{Brun:1073159} transport code are used to estimate $N_{\rm{ch}}$ from $N^{\rm{corr}}_{\rm{tracklets}}$. A second order polynomial correlation is assumed between these two quantities for the full $N^{\rm{corr}}_{\rm{tracklets}}$ interval, to obtain \dnchdeta. 

Several sources of systematic uncertainty were taken into account for the estimation of \dnchdeta. Possible deviations from the second order polynomial correlation were estimated by using other functions to quantify the $N_{\rm{ch}}$ from $N^{\rm{corr}}_{\rm{tracklets}}$ correlation, with   $\sim 5\%$ in all  multiplicity intervals in both pp and \pPb collisions.
%{\color{purple} with a value of 5\% value in all  multiplicity intervals in pp collisions.} \sout{, with values ranging from {\color{red}xx\% to yy\%} at thelowest (highest) multiplicity intervals}. 
The systematic uncertainty on the residual $z_{vtx}$ dependence due to differences between data and MC amounts to  $\sim$ 1\% in pp collisions and negligible in \pPb collisions. 

The average charged-particle pseudorapidity density $\left<\dnchdeta\right>$ for $\rm{INEL}>0$ events in pp collisions and in \pPb collisions were obtained to be  6.7 and 21.37, respectively. The $\rm{INEL}>0$ event class contains all events with at least 1 charged particle within $|\eta| < $1.0.  These values were cross-checked with published ALICE measurement~\cite{Adam:2015pza,Acharya:2020pit,Acharya:2020giw}, and were found to be in consistent. The normalized charged-particle pseudorapidity density $\left( \dnchdeta/\left<\dnchdeta\right> \right)$ in each event class considered is then calculated.  The resulting values of the normalized multiplicity for the event classes considered in the analysis are summarised in Table~\ref{Table:dndchValues}.

\begin{table}[!ht]
\caption{Average normalized charged-particle pseudorapidity density $\left( \dnchdeta/\left<\dnchdeta\right> \right)$ in $|\eta| < 1.0$ for each event class selected in pp and \pPb collisions.}
\label{Table:dndchValues}
  \begin{tabular*}{\textwidth}{@{\extracolsep{\fill}} c|cc|cc}
    \toprule
  \multicolumn{1}{c}{ } &
   
   \multicolumn{2}{c|}{pp \sqrt{s} = 13 \TeV} & \multicolumn{2}{c}{\pPb \sqrt{s_{\rm NN}} = 8.16 \TeV} \\
 \midrule

 \multicolumn{1}{c|}{Multiplicity class} &
     \multicolumn{1}{c}{$N^{\rm{corr}}_{\rm{tracklets}}$} & \multicolumn{1}{c|}{$\dnchdeta/\left<\dnchdeta\right>$} &  \multicolumn{1}{c}{$N^{\rm{corr}}_{\rm{tracklets}}$} & \multicolumn{1}{c}{$\dnchdeta/\left<\dnchdeta\right>$} \\
  \midrule    
     \multicolumn{1}{c|}{I} &
     \multicolumn{1}{c}{1-14} & \multicolumn{1}{c|}{0.48} &  \multicolumn{1}{c}{1-38} & \multicolumn{1}{c}{0.55} \\
       \midrule    
     \multicolumn{1}{c|}{II} &
     \multicolumn{1}{c}{15-24} & \multicolumn{1}{c|}{1.63} &  \multicolumn{1}{c}{39-55} & \multicolumn{1}{c}{1.36} \\
       \midrule    
     \multicolumn{1}{c|}{III} &
     \multicolumn{1}{c}{25-34} & \multicolumn{1}{c|}{2.50} &  \multicolumn{1}{c}{56-95} & \multicolumn{1}{c}{2.07} \\
       \midrule    
     \multicolumn{1}{c|}{IV} &
     \multicolumn{1}{c}{35-44} & \multicolumn{1}{c|}{3.34} &  \multicolumn{1}{c}{96-121} & \multicolumn{1}{c}{3.05} \\
       \midrule    
     \multicolumn{1}{c|}{V} &
     \multicolumn{1}{c}{45-54} & \multicolumn{1}{c|}{4.16} &  \multicolumn{1}{c}{122-300} & \multicolumn{1}{c}{3.89} \\
       \midrule    
     \multicolumn{1}{c|}{VI} &
     \multicolumn{1}{c}{55-64} & \multicolumn{1}{c|}{4.97} &  \multicolumn{1}{c}{} & \multicolumn{1}{c}{} \\
       \midrule    
     \multicolumn{1}{c|}{VII} &
     \multicolumn{1}{c}{65-120} & \multicolumn{1}{c|}{6.05} &  \multicolumn{1}{c}{} & \multicolumn{1}{c}{} \\
   \bottomrule
  \end{tabular*}
\end{table}